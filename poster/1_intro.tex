 近年、Deep learning(DL)を用いた画像解析技術の発展は目覚ましいものがあり、病理組織画像を対象とした研究も大きな成果を挙げている。その一方で、臨床診断の現場において実際にその技術と成果物を使用する機会は極めて少ない。病理組織画像に対するDLによる推論結果を得るために、いくつかの段階を経る必要があるためである。まず、病理組織を撮影し、DLによってその画像を解析し、その結果を確認する。この工程には、画像ファイルを外部ストレージに持ち出したり、鏡検室と物理的には離れたDLシステムへ出歩くことも含まれるかもしれない。すぐれたDLSが存在しても、これらのような手間ゆえに臨床現場と直接結びにくくなっている。\par
\vspace{0.5zh}
 本研究ではRaspberry Piを用いて臨床診断の行われる顕微鏡とDLの動作環境をシームレスに結びつける仕組みを提案する。\par
